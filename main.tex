\documentclass[12pt,a4paper,twoside]{book}

\usepackage{enumerate}
\usepackage{xcolor}
\usepackage{graphicx}
\usepackage{enumitem}
\setlist{nosep}
\usepackage{soul}
\usepackage{hyperref}
\usepackage{listings}
\usepackage{multicol}

\usepackage{url}

\newcommand{\cfbox}[2]{%
    \colorlet{currentcolor}{.}%
    {\color{#1}%
    \fbox{\color{currentcolor}#2}}%
}


\title{Creating EndSummerCamp: the book}
\author{sbiriguda ono? \\
        }

%\setcounter{secnumdepth}{5} % seting level of numbering (default for "report" is 3). With ''-1'' you have non number also for chapters
 %\setcounter{tocdepth}{5} % if you want all the levels in your table of contents

\begin{document}
\maketitle
\tableofcontents

\chapter{Intro: come leggere questo documento}
Questo documento serve un po' come linea guida per capire c

Per ogni task sarebbe opportuno avere: chi, cosa, come, quanto, quando, etc..

L'ideale sarebbe che un gruppo di completi sconosciuti, con questo documento, sia in grado di organizzare ESC in completa autonomia. 

Ho cercato di matenere un ordine cronologico degli eventi, evidenziando quelli che sono le criticit\`a da risolvere il prima possibile.

Sarebbe ideale avere, per ogni sezione e sottosezione un paragrafo che spiega come \`e possibile migliorare il task.

\chapter{Marzo-Aprile-Maggio}

\section{call for tendoni}
da decidere entro la fine di aprile

\section{call for Paninari}
con quanto anticipo
chi l'ha fatto 
stima prezzo



\section{presentazione hackinbo}
eh..


\section{eventbrite}
chi l'ha fatto
con quanto anticipo?
dipendenze: eventbrite va fatto PRIMA di hackinbo



\section{assicurazione}

\section{website}
ariele tvb 

\section{wiki}


\section{sponsor}



\chapter{3 months before ESC}

\section{isp}
con quanto anticipo
chi l'ha fatto 
stima prezzo

\section{birra}
quanta è stata consumata?
quanto tempo prima?
chi ha fatto il task?

\section{layout camp}
dopo aver visto i contents

\section{magliette}


\section{materiale}
\begin{itemize}
\item contents:
\begin{itemize}
\item due videoproiettori
\item puntatori laser per slide
\item adattatori hdmi to vga o dvi
\item telecomandi per avanzamento slide
\end{itemize}

\item antani
\end{itemize}

\section{press}

\subsection{comunicato stampa}
beh, visto che non si \`e usato il comunicato stampa di esc2016, direi che ne abbiamo uno nuovo per esc2017 :D

\subsection{comunicazioni con la stampa}
MarcoC

\subsection{conferenza stampa}
UBIiiiiiiiiiiiiiiiiiiiiiiiiiiiiiiiiiii

\chapter{Squatting the place: days before}
Quanta gente serve? Chi c'è di solito?

\subsection{showers}
durata montaggio
chi lo fa

\subsection{internet}

\subsection{corrente}
durata montaggio
chi lo fa

\subsection{bar}
durata montaggio
chi lo fa

\subsection{gazebi}
durata montaggio
chi lo fa

\subsection{tendoni}
durata montaggio
chi lo fa


\chapter{ESC has started}




\section{contents during ESC}
chi l'ha fatto 
stima prezzo: birra diocane
che materiali



\section{pulizie}
con quanto anticipo
chi l'ha fatto 
stima prezzo
che materiali

\section{login}

\section{gestione volontari}

\section{physical security}

\include{postesc}

% other (magari un chapter unico)

\chapter{other}


\section{Accounts}

\begin{itemize}
\item Twitter (endsummercamp)
\item Gmail (endsummercampvce) per stampare le cose al login e per scaricare le email delle slide dei reatori
\end{itemize}

\section{communication}


\subsection{engelsystem}

\subsection{telegram group}


\section{CTF}

\subsection{BOH}
Del cittieffe se ne è sempre occupato swappage, e così mi auguro sarà sempre.

\section{Idee ESC 2017}
C'è un bel thread in ML antani-gov sulle idee per il prossimo ESC. 
%TODO qualcuno lo riassume?

\include{press}

\end{document}
